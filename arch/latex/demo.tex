% zathura as the default pdf reader
% \documentclass{article}
\documentclass{beamer}

\title[short title]{Slide Shows in {\LaTeX} with Beamer}
\subtitle{Subtitle here}
\author{xc}
\institute{\large \textbf{Institute}: \\[6pt] word before is large and bold}
% \date{}

\usetheme[progressbar=frametitle]{metropolis}
% \usetheme{Warsaw}
% \usetheme{Frankfurt}
% \usetheme{Madrid}
\setbeamertemplate{frame numbering}[fraction]
\useoutertheme{metropolis}
\useinnertheme{metropolis}
\usefonttheme{metropolis}
\usecolortheme{spruce}
\setbeamercolor{background canvas}{bg=white}

% \usecolortheme{crane}
\definecolor{mygreen}{rgb}{.125,.5,.25}
\usecolortheme[named=mygreen]{structure}

\begin{document}
% add block fill in color
\metroset{block=fill}

\maketitle

\section{Section1}

\begin{frame}
    \frametitle{Test Title1}
    % keep title page in the frame
    \titlepage
    \begin{itemize}
        \item frame1-item1\pause
        \item frame1-item2\pause
        \item frame1-item3
    \end{itemize}
\end{frame}

\section{Section2}

\begin{frame}[t]{Test Title2}\vspace{20pt}
    test item\pause
    \begin{enumerate}
        \item frame2-item1\pause
        \item frame2-item2\pause
        \item frame2-item3\pause
    \end{enumerate}
    \vspace{0.5em}
    $\sqrt{x^2}=$\\[10pt] % $$ means math mode
    \begin{enumerate}[(A)]
        \item $frame2-item1$
        \item $frame2-item2$
        \item $frame2-item3$
    \end{enumerate}
\end{frame}

\section{Section3}

\begin{frame}{Test Title3 with Columns}
    \begin{columns}
        \column{.5\textwidth}
        \includegraphics[width=.85\textwidth]{Yugor.jpg}
        \column{.5\textwidth}
        test text
    \end{columns}
\end{frame}

\section{Section4}

\begin{frame}{Test Title4 with Columns}
    \begin{block}{definition of a block}
        % vspace means vertical space
        \vspace{0.5em}
        \includegraphics[width=.35\textwidth]{Yugor.jpg}
        \\[2pt]\textbf{bold} $italic$ \\[10pt] %end the line, 10pt of space
        draw an underline
        % \line(1,0){50} % just the underline without need of placing text
        \only<1>{\line(1,0){50}}
        \only<2->{\textcolor{magenta}{answer is here}}
        \,end underline.\\[10pt] %end the line, 10pt of space

        draw an underline
        \only<1-2>{\line(1,0){50}}
        \only<3->{\textcolor{magenta}{answer is here}}
        \,end underline.\\[10pt] %end the line, 10pt of space
        \vspace{0.5em}
    \end{block}
\end{frame}

\end{document}
